% Copyright 2019 Clara Eleonore Pavillet

% Author: Clara Eleonore Pavillet (original author), Leonard Quentin Marcq (modified for Tsinghua), 赵宗义 (参与了修改)
% Description: This is an unofficial Tsinghua University Beamer Template I made from scratch. Feel free to use it, modify it, share it.
% Version: 1.1


\documentclass{beamer}
\usepackage{ctex}
\usepackage{fontspec}
\input{Theme/Packages.tex}
\usetheme{workhw}

%%%<<<---
\newcommand{\variable}{\mathrm}
\newcommand{\field}{\mathsf}
%%%--->>>

\title{容器集群降本增效}
%\titlegraphic{\includegraphics[width=2cm]{Theme/Logos/tsinghua_emblem_dark.png}}
\author{刘欣}
%\institute{清华大学计算机科学与技术系}
\date{\today}

\begin{document}

{\setbeamertemplate{footline}{} 
\frame{\titlepage}}



\begin{frame}{主要内容}
\begin{itemize}
  \item \textbf{集群调度的核心挑战}
  \newline
%\begin{enumerate}
%\item 流记录项的分类和聚合
%\item 流记录项的存储和分析
%\end{enumerate}
  \item \textbf{集群调度技术演进}
  \newline
   \item \textbf{动态负载均衡服务落地实践}
  \newline
  \item \textbf{集群调度未来潜在演进方向}
  \newline
\end{itemize}
\end{frame}

\begin{frame}{集群调度的核心挑战}
\textbf{目标}
\begin{enumerate}
\item 整合离散和异构的服务器到弹性资源池,提高数据中心资源利用率,并提供自动化的运维能力;
\item 通过网络和存储系统,可以根据业务需求动态地分配计算实例,从而降低整体计算成本。
\end{enumerate}

\textbf{例如:} 
\newline
\setlength\parindent{24pt} \indent 按需为业务分配计算实例,并通过集群的规模效应降低计算成本

\end{frame}

\begin{frame}{集群调度的核心挑战}
提高在线集群资源利用率\&同时保障应用服务质量

\textbf{集群调度的复杂性}
\begin{enumerate}
\item 所有业务容器都需要在集群调度系统的管控下运行,如何保证整个应用的服务质量
\item \textbf{在保证服务质量的同时提高资源利用率,降低成本}
\item 满足不同业务个性化的调度需求
\item 有状态服务的云原生改造
\end{enumerate}

\end{frame}

\begin{frame}{集群调度的核心挑战}
\textbf{行业资源利用率水平}

据不完全统计,2020 年全球数据中心的服务器总量达到 1800 万台,并且正以每年 100 万台的速度增长。 统计数据表明,目前全球数据中心资源利用率仅为 10\% - 20\%,如此低的资源利用率意味着数据中心大量的资源浪费,导致目前数据中心的成本效率极低。

\end{frame}

\begin{frame}{集群调度技术演进}
\begin{enumerate}
\item 数据平面包含了两个哈希表, 分别是主表 (Main Table) 和辅表 (Ancillary Table).
\item 主表中每个哈希桶包含两个域: $\field{fieldID}$和$\field{count}$
\item 辅表中每个哈希桶也包含两个域: $\field{digest}$和$\field{count}$
\end{enumerate}
\begin{figure}
	\centering
	\includegraphics[width=0.6\linewidth]{figures/representation/datastructure}
\end{figure}

\end{frame}

\begin{frame}{动态负载均衡服务落地实践}
\begin{enumerate}
\item 数据平面包含了两个哈希表, 分别是主表 (Main Table) 和辅表 (Ancillary Table).
\item 主表中每个哈希桶包含两个域: $\field{fieldID}$和$\field{count}$
\item 辅表中每个哈希桶也包含两个域: $\field{digest}$和$\field{count}$
\end{enumerate}
\begin{figure}
	\centering
	\includegraphics[width=0.6\linewidth]{figures/representation/datastructure}
\end{figure}

\end{frame}

\begin{frame}{集群调度未来潜在演进方向}
\begin{enumerate}
\item 数据平面包含了两个哈希表, 分别是主表 (Main Table) 和辅表 (Ancillary Table).
\item 主表中每个哈希桶包含两个域: $\field{fieldID}$和$\field{count}$
\item 辅表中每个哈希桶也包含两个域: $\field{digest}$和$\field{count}$
\end{enumerate}
\begin{figure}
	\centering
	\includegraphics[width=0.6\linewidth]{figures/representation/datastructure}
\end{figure}

\end{frame}

  
\begin{frame}{}
\begin{center}
    \textbf{谢谢!}
\end{center}
\end{frame}


\end{document}

